%-----------------------------------------------------------------------------------------------------------------------------------------------%
%	The MIT License (MIT)
%
%	Copyright (c) 2021 Jitin Nair
%
%	Permission is hereby granted, free of charge, to any person obtaining a copy
%	of this software and associated documentation files (the "Software"), to deal
%	in the Software without restriction, including without limitation the rights
%	to use, copy, modify, merge, publish, distribute, sublicense, and/or sell
%	copies of the Software, and to permit persons to whom the Software is
%	furnished to do so, subject to the following conditions:
%	
%	THE SOFTWARE IS PROVIDED "AS IS", WITHOUT WARRANTY OF ANY KIND, EXPRESS OR
%	IMPLIED, INCLUDING BUT NOT LIMITED TO THE WARRANTIES OF MERCHANTABILITY,
%	FITNESS FOR A PARTICULAR PURPOSE AND NONINFRINGEMENT. IN NO EVENT SHALL THE
%	AUTHORS OR COPYRIGHT HOLDERS BE LIABLE FOR ANY CLAIM, DAMAGES OR OTHER
%	LIABILITY, WHETHER IN AN ACTION OF CONTRACT, TORT OR OTHERWISE, ARISING FROM,
%	OUT OF OR IN CONNECTION WITH THE SOFTWARE OR THE USE OR OTHER DEALINGS IN
%	THE SOFTWARE.
%	
%
%-----------------------------------------------------------------------------------------------------------------------------------------------%

%----------------------------------------------------------------------------------------
%	DOCUMENT DEFINITION
%----------------------------------------------------------------------------------------

% article class because we want to fully customize the page and not use a cv template
\documentclass[a4paper,12pt]{article}

%----------------------------------------------------------------------------------------
%	FONT
%----------------------------------------------------------------------------------------

% % fontspec allows you to use TTF/OTF fonts directly
% \usepackage{fontspec}
% \defaultfontfeatures{Ligatures=TeX}

% % modified for ShareLaTeX use
% \setmainfont[
% SmallCapsFont = Fontin-SmallCaps.otf,
% BoldFont = Fontin-Bold.otf,
% ItalicFont = Fontin-Italic.otf
% ]
% {Fontin.otf}

%----------------------------------------------------------------------------------------
%	PACKAGES
%----------------------------------------------------------------------------------------
\usepackage{url}
\usepackage{parskip} 	

%other packages for formatting
\RequirePackage{color}
\RequirePackage{graphicx}
\usepackage[usenames,dvipsnames]{xcolor}
\usepackage[scale=0.9]{geometry}

%tabularx environment
\usepackage{tabularx}

%for lists within experience section
\usepackage{enumitem}

% centered version of 'X' col. type
\newcolumntype{C}{>{\centering\arraybackslash}X} 

%to prevent spillover of tabular into next pages
\usepackage{supertabular}
\usepackage{tabularx}
\newlength{\fullcollw}
\setlength{\fullcollw}{0.47\textwidth}

%custom \section
\usepackage{titlesec}				
\usepackage{multicol}
\usepackage{multirow}

%CV Sections inspired by: 
%http://stefano.italians.nl/archives/26
\titleformat{\section}{\Large\scshape\raggedright}{}{0em}{}[\titlerule]
\titlespacing{\section}{0pt}{10pt}{10pt}

%for publications
\usepackage[style=authoryear,sorting=ynt, maxbibnames=2]{biblatex}

%Setup hyperref package, and colours for links
\usepackage[unicode, draft=false]{hyperref}
\definecolor{linkcolour}{rgb}{0,0.2,0.6}
\hypersetup{colorlinks,breaklinks,urlcolor=linkcolour,linkcolor=linkcolour}
\addbibresource{citations.bib}
\setlength\bibitemsep{1em}

%for social icons
\usepackage{fontawesome5}

%debug page outer frames
%\usepackage{showframe}

%----------------------------------------------------------------------------------------
%	BEGIN DOCUMENT
%----------------------------------------------------------------------------------------
\begin{document}

% non-numbered pages
\pagestyle{empty} 

%----------------------------------------------------------------------------------------
%	TITLE
%----------------------------------------------------------------------------------------

% \begin{tabularx}{\linewidth}{ @{}X X@{} }
% \huge{Your Name}\vspace{2pt} & \hfill \emoji{incoming-envelope} email@email.com \\
% \raisebox{-0.05\height}\faGithub\ username \ | \
% \raisebox{-0.00\height}\faLinkedin\ username \ | \ \raisebox{-0.05\height}\faGlobe \ mysite.com  & \hfill \emoji{calling} number
% \end{tabularx}

\begin{tabularx}{\linewidth}{@{} C @{}}
\huge{Daniel Nathan Reis Silva} \\[7.5pt]
\href{https://github.com/danielnatham}{\raisebox{-0.05\height}\faGithub\ danielnatham} \ $|$ \ 
\href{https://www.linkedin.com/in/reis-silva-daniel/}{\raisebox{-0.05\height}\faLinkedin\ reis-silva-daniel} \ $|$ \ 
\href{mailto:danielnatham@gmail.com}{\raisebox{-0.05\height}\faEnvelope \ danielnatham@gmail.com} \\
\\

\end{tabularx}

%----------------------------------------------------------------------------------------
% EXPERIENCE SECTIONS
%----------------------------------------------------------------------------------------
%Interests/ Keywords/ Summary

\section{Resumo}
Sou um \textbf{Desenvolvedor de Software FullStack} com mais de 4 anos de experiência e graduando em \textbf{Ciência da Computação} pela UFS. Atuei em projetos de grande porte nos setores de \textbf{varejo, saúde, educação e fiscal}, com foco em entregar soluções escaláveis, seguras e de alto impacto.

Possuo experiência sólida em \textbf{backend} (Java, Spring Boot, PHP/Laravel, Go, Python, Django) e \textbf{frontend} (React, Angular, SenchaJS, jQuery), além de integração com sistemas fiscais (NF-e, NFC-e) e ERPs. Domino bancos de dados relacionais (\textbf{PostgreSQL, Oracle, MySQL}) e práticas modernas de \textbf{DevOps} com Docker, CI/CD e Nginx.

Tenho histórico em \textbf{migração de sistemas legados}, otimização de desempenho de queries SQL e manutenção evolutiva de arquiteturas distribuídas. Participo ativamente de todas as fases do ciclo de desenvolvimento — requisitos, implementação, testes e deploy.

Além da vivência técnica, destaco competências em \textbf{trabalho em equipe ágil (Scrum)}, liderança técnica e resolução de problemas. Certificado pelo \textbf{Google Cybersecurity Professional Certificate} e com \textbf{inglês B2 (TOEFL ITP)}, busco sempre aliar conhecimento técnico a boas práticas de arquitetura e segurança da informação.

%----------------------------------------------------------------------------------------
% Experience
%----------------------------------------------------------------------------------------
\section{Experiências}

\begin{tabularx}{\linewidth}{ @{}l r@{} }
\textbf{Full Stack Developer} & \hfill Dez. 2024 - atualmente \\
\textbf{- CD2 Retail Tech}  & \hfill Remoto\\[3.75pt]
\multicolumn{2}{@{}X@{}}{
    \begin{minipage}[t]{\linewidth}
        \begin{itemize}[nosep,after=\strut, leftmargin=1em, itemsep=3pt]
            \item[--] Desenvolvimento e manutenção de novas funcionalidades para sistemas de gestão de lojas e PDV, melhorando eficiência e usabilidade;
            \item[--] Integração com ERPs (como Bluesoft) e sistemas fiscais via SOAP e REST APIs, otimizando fluxos de dados e reduzindo tempo de processamento;
            \item[--] Otimização de queries SQL, aumentando o desempenho de relatórios e processos internos, com redução significativa no tempo de geração;
            \item[--] Atualização de interfaces web e PDV com \textbf{SenchaJS} e \textbf{jQuery}, proporcionando experiência mais moderna e fluida;
            \item[--] Manutenção evolutiva em integrações SOAP de NF-e e NFC-e, reduzindo complexidade cognitiva e melhorando qualidade do código;
            \item[--] Tecnologias: \textbf{PHP, jQuery, Oracle, PostgreSQL, SenchaJS, SOAP, REST, JIRA}.
        \end{itemize}
    \end{minipage}
}
\end{tabularx}

\begin{tabularx}{\linewidth}{ @{}l r@{} }
\textbf{Back End Developer} & \hfill Dez. 2023 - Nov. 2024 \\
\textbf{- SergipeTec}  & \hfill Remoto\\[3.75pt]
\multicolumn{2}{@{}X@{}}{
    \begin{minipage}[t]{\linewidth}
        \begin{itemize}[nosep,after=\strut, leftmargin=1em, itemsep=3pt]
            \item[--] Migração de monólito Java 8 para arquitetura de microsserviços em \textbf{Java 17} e \textbf{Spring Boot}, garantindo escalabilidade e manutenibilidade;
            \item[--] Implementação de testes unitários e de integração, além de documentação de APIs com \textbf{Swagger};
            \item[--] Execução de tarefas utilizando metodologia \textbf{SCRUM} e implementação de pipelines \textbf{CI/CD} no GitLab;
            \item[--] Containerização de serviços com \textbf{Docker}, facilitando o deploy de ambientes;
            \item[--] Tecnologias: \textbf{Java 17, Spring Boot, Oracle, JUnit, Swagger, Postman, JSF, Docker}.
        \end{itemize}
    \end{minipage}
}
\end{tabularx}

\begin{tabularx}{\linewidth}{ @{}l r@{} }
\textbf{Full Stack Developer} & \hfill Jan. 2022 - Dez. 2023 \\
\textbf{- ZDoc - Document Management Technology}  & \hfill Aracaju, SE\\[3.75pt]
\multicolumn{2}{@{}X@{}}{
\begin{minipage}[t]{\linewidth}
    \begin{itemize}[nosep,after=\strut, leftmargin=1em, itemsep=3pt]
        \item[--] Manutenção de ERP legado com \textbf{HTML, CSS, PHP 7 e jQuery};
        \item[--] Migração do ERP para novas tecnologias utilizando \textbf{React} e \textbf{Laravel} integrados a \textbf{Oracle};
        \item[--] Gestão de tarefas, versionamento e deploy com \textbf{Trello} e \textbf{GitLab};
        \item[--] Cobertura de código com TDD utilizando a biblioteca \textbf{PHPUnit};
        \item[--] Tecnologias: \textbf{PHP, jQuery, Laravel, MySQL, React, REST, JIRA}.
    \end{itemize}
    \end{minipage}
}
\end{tabularx}

\begin{tabularx}{\linewidth}{ @{}l r@{} }
\textbf{Full Stack Developer} & \hfill Set. 2021 - Jan. 2022 \\
\textbf{- Pueri Health}  & \hfill Aracaju, SE\\[3.75pt]
\multicolumn{2}{@{}X@{}}{
\begin{minipage}[t]{\linewidth}
    \begin{itemize}[nosep,after=\strut, leftmargin=1em, itemsep=3pt]
        \item[--] Prototipação de solução \textbf{MVP} para programa de indução de startups;
        \item[--] Deploy da aplicação na plataforma em nuvem \textbf{Heroku};
        \item[--] Documentação de APIs com \textbf{Swagger} e cobertura de testes unitários com \textbf{JUnit};
        \item[--] Tecnologias: \textbf{Java 17, Spring Boot, Thymeleaf, Bootstrap, Swagger, Heroku, PostgreSQL}.
    \end{itemize}
    \end{minipage}
}
\end{tabularx}


%----------------------------------------------------------------------------------------
%	EDUCATION
%----------------------------------------------------------------------------------------
\section{Educação}
\begin{tabularx}{\linewidth}{lXr}	
2021 - 2027 & Bacharelado em Ciência da Computação pela \textbf{Universidade Federal de Sergipe (UFS)} & Aracaju-SE \\\\

2020 - 2022 & Formação Continuada para Programadores Web pelo \textbf{Instituto Federal da Paraíba (IFPB)} & Pedras de Fogo-PB \\ 
\end{tabularx}

%----------------------------------------------------------------------------------------
%  Habilidades e Competências
%----------------------------------------------------------------------------------------
\section{Habilidades e Competências}
\begin{tabularx}{\linewidth}{@{}l X@{}}
\multicolumn{2}{@{}X@{}}{
\begin{minipage}[t]{\linewidth}
    \begin{itemize}[nosep,after=\strut, leftmargin=1em, itemsep=3pt]
        \item[--] \textbf{Rápida aprendizagem}, adaptabilidade e capacidade de liderança técnica;
        \item[--] Experiência com \textbf{trabalho em equipe ágil} (Scrum) e ferramentas de gestão (\textbf{JIRA, Trello, GitLab});
        \item[--] \textbf{Linguagens de programação}: Java (Spring Boot, JUnit), PHP (Laravel, PHPUnit), JavaScript, TypeScript, Go, Python, C;
        \item[--] \textbf{Frontend}: React, Angular, SenchaJS, jQuery, Thymeleaf, Bootstrap, HTML e CSS;
        \item[--] \textbf{Backend}: Spring Boot, Django, Laravel, PHP 7, JSF, APIs REST e SOAP;
        \item[--] \textbf{Bancos de dados}: PostgreSQL, MySQL e Oracle, com otimização de queries SQL;
        \item[--] Experiência em \textbf{DevOps}: Docker, Docker Compose, Git, GitLab CI/CD, Nginx Proxy Manager, Heroku;
        \item[--] Conhecimentos em \textbf{arquitetura de microsserviços}, \textbf{arquitetura limpa} e boas práticas de código;
        \item[--] Integrações com \textbf{sistemas fiscais (NF-e, NFC-e)}, ERPs (Bluesoft) e serviços externos via APIs;
        \item[--] Certificação: \textbf{Google Cybersecurity Professional Certificate} (2025);
        \item[--] Idiomas: \textbf{Inglês B2} (TOEFL ITP, 2018).
    \end{itemize}
    \end{minipage}
}
\end{tabularx}

%----------------------------------------------------------------------------------------
%  Languages
%----------------------------------------------------------------------------------------
\section{Idiomas e Certificados}
\begin{tabularx}{\linewidth}{@{}l X@{}}
\textbf{Google Cybersecurity Professional Certificate} & \hfill Emitido pela Coursera, Jul. 2025 \\ 
\textbf{Linux Fundamentos} & \hfill Emitido pela FIAP, Dez. 2021 \\
\textbf{Inglês B2} & \hfill TOEFL ITP - Universidade Federal de Sergipe (UFS), Aracaju-SE, 2018 \\
\end{tabularx}


% %----------------------------------------------------------------------------------------
% % Projects
% %----------------------------------------------------------------------------------------
\section{Projetos}
\begin{tabularx}{\linewidth}{ @{}l r@{} }
\textbf{Hacktoberfest 2021} & \hfill \href{https://github.com/sicpjs/estrutura-e-interpretacao-de-programas-de-computador-javascript}{Tradução no GitHub} \\[3.75pt]
\multicolumn{2}{@{}X@{}}{
    Participação na tradução do livro \emph{"Estrutura e Interpretação de Programas de Computador em JavaScript"} do inglês para o português, como parte da comunidade open source no evento Hacktoberfest 2021.
}  \\
\end{tabularx}

\vfill
\center{\footnotesize Última Atualização: 10 de Setembro de 2025}

\end{document}